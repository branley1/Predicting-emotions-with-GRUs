\documentclass[11pt]{article}
\usepackage{fullpage}
\usepackage{graphicx}

\title{CS63 Spring 2024\\Final Project Checkpoint}
\author{Alex Huynh, Branley Mmasi}
\date{}

\begin{document}

\maketitle

\section{Project Goal}

For this poject, we want to use a Recurrent Neural Network (RNN) to assess
tweets for emotions. We will be using Kaggle's "Emotions" dataset to train
and test our model.

\section{AI Methods Used}

For this project, we will be building an RNN. We will use the Keras library
to help us build and train this model.

\section{Staged Development Plan}

\begin{enumerate}
    \item Research how an RNN works and look for dataset (Kaggle's Emotions)
    \item Prepare the dataset (split into train and test, tokenize)
    \item Use Keras library build an RNN, describing each steps with comments
    \item Train and test using the dataset
    \item Write code to help us visualize accuracy of model (confusion matrix, learning curves)
    \item Adjust hyperparameters and train until satisfied with accuracy (>90\%)
    \item Create a program that takes a piece of text and outputs an emotion
    \item Writeup and presentation
        \begin{enumerate}
            \item Describe what an RNN is, how it differs from other
            models we've learned in class, and why its useful
            \item Describe dataset
            \item Describe network (what the model is doing to each input)
            \item Assess model using accuracy, confusion matrix, and program
        \end{enumerate}
    \item Stretch Goals
        \begin{enumerate}
            \item Break acronyms/chat words into words (e.g. lol, btw, brb)
            \item See which words is associated to each emotions
            \item Begin building our own RNN
                \begin{enumerate}
                    \item Tokenization
                    \item Embedding
                    \item Network
                    \item Backpropation through time
                    \item Assess with toy dataset (like XOR but for text)
                \end{enumerate}
        \end{enumerate}

\end{enumerate}

\section{Measure of Success}

Our project is successful if we can complete our main goals. Ultimately,
our final product will be similar to the final product of our CNN lab with
the addition of our program that utilizes the network.


\section{Plans for Analyzing Results}

We will be using a confusion matrix to analyze the accuracy of our model,
and undestand where our model is likely to mess up. We will also use our
program to analyze our model's accuracy. We'll feed the program various
texts and keep track of how many times it accurately outputs an emotion,
as well as nonsense data to see what the model does.

\end{document}
